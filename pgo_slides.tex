\documentclass{beamer}
% \usepackage{forloop}
% \usepackage{pgffor}
\usepackage[pscoord]{eso-pic}
\usepackage{minted}

\usepackage{hyperref}
\hypersetup{colorlinks=true, linkcolor=black, citecolor=black, urlcolor=blue}

\usepackage{tikz}
\usetikzlibrary{positioning}

\let\Tiny=\tiny

\usefonttheme[onlymath]{serif}

\definecolor{bashkey}{HTML}{C4A000}
\definecolor{bashval}{HTML}{75507B}
\definecolor{bashvar}{HTML}{06989A}
\definecolor{termdir}{HTML}{729CCF}

\newcommand{\ttc}[2]{\textcolor{#1}{\texttt{#2}}}

\newcommand{\toplink}[1]{%
  \placetextbox{0.61}{0.985}{
    \href{#1}{\includegraphics[width=6mm]{fig/link.pdf}}}
}

\setbeamersize{text margin left=5px,text margin right=5px}
\setbeamertemplate{navigation symbols}{}
\setbeamertemplate{footline}{%
  \usebeamerfont{footline}\usebeamercolor[fg]{footline}%
  \insertauthor,\ \ \insertdate\ \ PGO
  \hfill%
  \insertframenumber/\inserttotalframenumber%
}

\newcommand{\placetextbox}[3]{% \placetextbox{<horizontal pos>}{<vertical pos>}{<stuff>}
  \setbox0=\hbox{#3}% Put <stuff> in a box
  \AddToShipoutPictureFG*{% Add <stuff> to current page foreground
    \put(\LenToUnit{#1\paperwidth},\LenToUnit{#2\paperheight}){\vtop{{\null}\makebox[0pt][c]{#3}}}%
  }%
}%

\makeatletter
\DeclareRobustCommand\njet{\@ifnextchar[{\@@njet}{\@njet}}
\def\@@njet[#1]{\ensuremath{N_\text{jet}^{\geq {#1}\text{GeV}}}}
\def\@njet{\ensuremath{N_\text{jet}}}
\makeatother

\addtobeamertemplate{frametitle}{}{%
  \begin{tikzpicture}[remember picture,overlay]
    \node[anchor=north east] at (current page.north east){
      \includegraphics[height=20pt]{fig/msu_helmet.pdf}
      \vspace{2px}
      \includegraphics[height=20pt]{fig/msu_text.pdf}
    };
  \end{tikzpicture}
}

% ===================================================================

\author{Sam Marinelli, Ivan Pogrebnyak}
\begin{document}
\title{Better Data Analysis with Modern \texttt{C++}\\ Know Your Tools}
\date{April 6, 2017}

% ===================================================================

\frame{
  \begin{tikzpicture}[remember picture,overlay]
    \node[anchor=north] at (current page.north){
      \includegraphics[height=40pt]{fig/msu_helmet.pdf}
      \vspace{2px}
      \includegraphics[height=40pt]{fig/msu_text.pdf}
    };
  \end{tikzpicture}
  \titlepage
}

% ===================================================================

\frame{\frametitle{Modern \texttt{C++}}
  \vspace{-5mm}
  \begin{itemize}
    \item More expressive
    \item Highly abstract
    \item Less verbose
    \item Runs faster
  \end{itemize}
    \vspace{3mm}
    \includegraphics[width=\textwidth]{fig/cave}
    \vspace{1mm}
    \vspace{-3mm}
  \begin{itemize}
    \item Working code from examples on GitHub:
    \begin{itemize}
      \item \raisebox{-.08\height}{\includegraphics[width=9px]{fig/github.pdf}}
            \url{https://github.com/ivankp/pgo_cxx}
    \end{itemize}
  \end{itemize}
}

% ===================================================================

\frame{\frametitle{Get a new compiler}
  % \toplink{http://mirrors.concertpass.com/gcc/releases/}
  \toplink{https://gcc.gnu.org/}
  \vspace{-3mm}
  \begin{itemize}
    \item \texttt{\ttc{bashkey}{wget} {\scriptsize\url{http://mirrors.concertpass.com/gcc/releases/gcc-6.3.0/gcc-6.3.0.tar.bz2}}}
    \item \texttt{\ttc{bashkey}{tar} xvjf gcc-6.3.0.tar.bz2}
    \item \texttt{\ttc{bashkey}{cd} gcc-6.3.0}
    \item \texttt{./contrib/download\_prerequisites}
    \item \texttt{\ttc{bashkey}{cd} ..}
    \item \texttt{\ttc{bashkey}{mkdir} build-gcc-6.3.0}
    \item \texttt{\ttc{bashkey}{cd} build-gcc-6.3.0}

    \item {\scriptsize\texttt{\ttc{bashval}{\$PWD}/../gcc-6.3.0/configure --prefix=\ttc{bashval}{\$HOME}/gcc-6.3.0 --enable-languages=c,c++,fortran --with-default-libstdcxx-abi=gcc4-compatible --enable-bootstrap --enable-threads=posix --with-long-double-128 --enable-long-long --enable-lto --enable-gnu-unique-object --enable-gold --with-system-zlib --disable-nls}}

    \item \texttt{\ttc{bashkey}{make} -j8}
    \item \texttt{\ttc{bashkey}{make} -j8 install}
  \end{itemize}
}

% ===================================================================
\newlength{\defaultlmii}
\setlength{\defaultlmii}{\leftmarginii}
\setlength{\leftmarginii}{12pt}

\frame{\frametitle{Get a new compiler}
  \begin{itemize}
    \item After installation is complete, you can remove \ttc{termdir}{build-gcc-6.3.0}.
    \item If you want to use the new compiler by default,\\
          only two environmental variables are needed
    \vspace{3mm}
    \begin{itemize}
      \item \texttt{\ttc{bashkey}{export} \ttc{bashvar}{PATH}=\ttc{bashval}{\$HOME}/gcc-6.3.0/bin:\ttc{bashval}{\$PATH}}
      \item \texttt{\ttc{bashkey}{export} \ttc{bashvar}{LD\_LIBRARY\_PATH}=\ttc{bashval}{\$HOME}/gcc-6.3.0/lib64:\ttc{bashval}{\$LD\_LIBRARY\_PATH}}
    \end{itemize}
    \vspace{3mm}
    \item Or you can just call the \texttt{g++} executable by the whole path.
    \item New headers and libraries are automatically looked up first during
      compilation and linking. No extra environmental variables needed.
    \item Setting \ttc{bashvar}{LD\_LIBRARY\_PATH} can be avoided by compiling
      with \texttt{-Wl,-rpath=\ttc{bashval}{\$HOME}/gcc-6.3.0/lib64}
  \end{itemize}
}

\setlength{\leftmarginii}{\defaultlmii}
% ===================================================================

\frame{\frametitle{$\lambda$ functions}
  \toplink{http://en.cppreference.com/w/cpp/language/lambda}

}

% ===================================================================
% https://www.sharelatex.com/learn/Code_Highlighting_with_minted

\begin{frame}[fragile=singleslide]
\frametitle{Variadic templates}
  \toplink{http://en.cppreference.com/w/cpp/language/parameter_pack}

\vspace{-5mm}
{\large$$\sum_ix_i^2$$}

\vspace{-5mm}
\begin{minted}[mathescape,fontsize=\footnotesize,frame=lines]{cpp}
template <typename T>
constexpr T sq(T x) noexcept { return x*x; }

template <typename T, typename... TT>
constexpr T sq(T x, TT... xx) noexcept { return sq(x)+sq(xx...); }
\end{minted}

\vspace{4mm}
Usage example:

\vspace{-4mm}
\begin{minted}[mathescape,fontsize=\footnotesize,frame=lines]{cpp}
double hypotenuse = std::sqrt(sq(3.,4.)); // 5
\end{minted}
\end{frame}

% ===================================================================

\begin{frame}[fragile=singleslide]
\frametitle{Variadic templates}
  \toplink{http://en.cppreference.com/w/cpp/language/parameter_pack}

\vspace{-5mm}
\begin{minted}[mathescape,fontsize=\footnotesize,frame=lines]{cpp}
#include <sstream>

template <typename S, typename T>
inline void cat_impl(S& ss, T&& x) {
  ss << std::forward<T>(x);
}

template <typename S, typename T, typename... TT>
inline void cat_impl(S& ss, T&& x, TT&&... xx) {
  ss << std::forward<T>(x);
  cat_impl(ss,std::forward<TT>(xx)...);
}

template <typename... TT>
inline std::string cat(TT&&... xx) {
  std::stringstream ss;
  cat_impl(ss,std::forward<TT>(xx)...);
  return ss.str();
}
\end{minted}
\end{frame}

% ===================================================================

\begin{frame}[fragile=singleslide]
\frametitle{Variadic templates}
  \toplink{http://en.cppreference.com/w/cpp/language/parameter_pack}

\vspace{-5mm}
Example 1:
\vspace{-3mm}
\begin{minted}[mathescape,fontsize=\footnotesize,frame=lines]{cpp}
cout << cat("char_array",' ',5,' ',std::fixed,std::setprecision(2),4.2)
     << endl; // char_array 5 4.20
\end{minted}

\vspace{3mm}
Example 2:
\vspace{-3mm}
\begin{minted}[mathescape,fontsize=\footnotesize,frame=lines]{cpp}
TFile file("file.root");
std::string name("jet2_pT");
TH1 *hist = dynamic_cast<TH1*>(file->Get(name.c_str()));

if (!hist) throw std::runtime_error(cat(
  name," histogram does not exists in file ",file.GetName()));

if (hist->GetEntries()==0) throw std::runtime_error(cat(
  hist->GetName()," histogram is empty"));
\end{minted}
\end{frame}

% ===================================================================

\begin{frame}[fragile=singleslide]
\frametitle{\texttt{C++17} fold expressions}
  \toplink{http://en.cppreference.com/w/cpp/language/fold}

\vspace{-5mm}
\begin{minted}[mathescape,fontsize=\footnotesize,frame=lines]{cpp}
template <typename... Args>
inline std::string cat(Args&&... args) {
  std::stringstream ss;
  (ss << ... << args); // fold expression
  return ss.str();
}
\end{minted}
\end{frame}

% ===================================================================

\end{document}
